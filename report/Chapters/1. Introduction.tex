% ---------------------------------
% Introduction
% ---------------------------------
\section{Introduction}
Write a brief introduction. Explain what this experiment is for and why researchers use it. 


% ---------------------------------
% Objectives
% ---------------------------------
\section{Objectives} \label{section: introduction}
Write your experiment objectives here. You can use bulleted points if you like.

\begin{itemize}
    \item Objective 1
    \item Objective 2
\end{itemize}

% ---------------------------------
% Equipment Required
% ---------------------------------
\section{Equipment Required} 
Write the pieces of equipment required for the experiment.
Equipment 1, Equipment 2, Equipment 3, and Equipment 4.

% ---------------------------------
% Methodology
% ---------------------------------
\section{Methodology}
Write down the step-by-step methodology. You can add a bulleted list or flowchart (\cref{fig:flowchart}).

\begin{itemize}
    \item Item 1
    \item Item 2
    \begin{itemize}
        \item subitem 1
        \begin{itemize}
            \item subsubitem 1
            \item subsubitem 2
        \end{itemize}
    \end{itemize}
    \item Item 3
\end{itemize}

% Add flowchart
% This flow chart is created by the author

% adjustbox is used to limit the figure inside the page
% -- means normal arrow
%  -| horizontal followed by the vertical arrow
%  |- vertical followed by the horizontal arrow


\begin{figure}[H]

    \begin{center}
        \begin{adjustbox}{max height=0.7\textheight, center, width=0.5\textwidth}
            \begin{tikzpicture}[node distance=2.5cm]
                \node (step1_1) [startstop] {\textbf{\Large Innulla risus}};
                \node (step1_2) [process, below of = step1_1] {\textbf{\Large Lorem ipsum}};
                \node (step1_3) [process, below of = step1_2] {\textbf{\Large Suspendisse potenti}};
                \node (step1_4) [process, below of = step1_3] {\textbf{\Large Morbi odio velit}};

% ---------------------------------------------
% vertical (primary) Division
% ---------------------------------------------
                % Flow

                \node (step1_5) [process, below of = step1_4, yshift = 0cm] {\textbf{\Large Suspendisse a}};

                \node (step1_6) [process, below of = step1_5, yshift = 0cm] {\textbf{\Large Maecenas velit lectus}};

                \node (step1_7) [process, below of = step1_6, yshift = 0cm] {\textbf{\Large Suspendisse a}};

                \node (step1_8) [process, below of = step1_7, yshift = 0cm] {\textbf{\Large Pellentesque pulvinar}};

                \node (step1_9) [process, below of = step1_8, yshift = 0cm] {\textbf{\Large Nam ullamcorper}};

                \node (step1_10) [process, below of = step1_9, yshift = 0cm] {\textbf{\Large Phasellus}};


% ---------------------------------------------
% All arrows (for vertical column
% ---------------------------------------------

                %  All arrows (for vertical column
                \draw [arrow] (step1_1) -- (step1_2);
                \draw [arrow] (step1_2) -- (step1_3);
                \draw [arrow] (step1_3) -- (step1_4);
                \draw [arrow] (step1_4) -- (step1_5);
                \draw [arrow] (step1_5) -- (step1_6);
                \draw [arrow] (step1_6) -- (step1_7);
                \draw [arrow] (step1_7) -- (step1_8);
                \draw [arrow] (step1_8) -- (step1_9);
                \draw [arrow] (step1_9) -- (step1_10);

% ---------------------------------------------
% Left Division
% --------------------------------------------- 
                \node (step2_1) [process, left of = step1_4, xshift=-8cm, yshift=-1.4cm] {\textbf{\Large Cras sem felis Cras sem felis}};

                \node (step2_2) [process, below of = step2_1, yshift=0cm] {\textbf{\Large Pellentesque pulvinar}};

                \node (step2_3) [process, below of = step2_2, yshift=0cm] {\textbf{\Large Suspendisse potenti}};
                

% ---------------------------------------------
% All arrows (for the left column)
% ---------------------------------------------

                %  All arrows (for vertical column
                \draw [arrow] (step1_4) -| (step2_1);
                \draw [arrow] (step2_1) -- (step2_2);
                \draw [arrow] (step2_2) -- (step2_3);
                \draw [arrow] (step2_3) |- (step1_7);

% ---------------------------------------------
% Right Division
% ---------------------------------------------   
                \node (step3_1) [process, right of = step1_7, xshift=8cm, yshift=-3.9cm] {\textbf{\Large Suspendisse potenti}};

% ---------------------------------------------
% All arrows (for right column)
% ---------------------------------------------

                %  All arrows (for vertical column
                \draw [arrow] (step1_7) -| (step3_1);
                \draw [arrow] (step3_1) |- (step1_10);
                
            \end{tikzpicture}
        \end{adjustbox}
    \end{center}
    \caption{Methodology.}
    \label{fig:flowchart}
\end{figure}

% ---------------------------------
% Procedure
% ---------------------------------
\section{Procedure}
In this section, you need to explain how you conducted the experiment. This could be done with figures (example \cref{fig: three images}) or observation tables (example \cref{table: style 1}). The figure and table number follows the following format \texttt{<section number>.<figure number>}.

You can also add citations of past work or experiments conducted by scientists \cite{kottwitz2011}.

% Table with style 1
\begin{table}[ht]
\rowcolors{2}{SIT-red!10}{white}
\centering
\caption{A table without vertical lines.}
\begin{tabular}[t]{ccccc}
\toprule
\color{SIT-red}\textbf{Column 1}&\color{SIT-red}\textbf{Column 2}&\color{SIT-red}\textbf{Column 3}&\color{SIT-red}\textbf{Column 4}&\color{SIT-red}\textbf{Column 5}\\
\midrule
Entry 1&1&2&3&4\\
Entry 2&1&2&3&4\\
Entry 3&1&2&3&4\\
Entry 4&1&2&3&4\\
\bottomrule
\end{tabular}
\label{table: style 1}
\end{table}

% Figure with three images
\begin{figure}[H]
     \centering
     \begin{subfigure}[b]{0.3\textwidth}
         \centering
         \includegraphics[width=\textwidth]{example-image-a}
         \caption{image a}
         \label{fig: style 1 image a}
     \end{subfigure}
     \hfill
     \begin{subfigure}[b]{0.3\textwidth}
         \centering
         \includegraphics[width=\textwidth]{example-image-b}
         \caption{image b}
         \label{fig: style 1 image b}
     \end{subfigure}
     \hfill
     \begin{subfigure}[b]{0.3\textwidth}
         \centering
         \includegraphics[width=\textwidth]{example-image-c}
         \caption{image c}
         \label{fig: style 1 image c}
     \end{subfigure}
        \caption{Three images}
        \label{fig: three images}
\end{figure}

% ---------------------------------
% Calculations and Analysis
% ---------------------------------

\section{Calculations and Analysis}

This section explains the data analysis or exploratory analysis as per the experimental procedure.  


% Sub Section Heading
\subsection{Sub Section Heading}
This is a subsection under the main section.

\begin{enumerate}
    \item Item 1
    \item Item 2
    \begin{enumerate}
        \item subitem 1
        \begin{enumerate}
            \item subsubitem 1
            \item subsubitem 2
        \end{enumerate}
    \end{enumerate}
    \item Item 3
\end{enumerate}

\subsection{Sub Section Heading}
This is another subsection under the section.\\

Here few equation styles that you can use.\\

% Equation
One-line \cref{Eq1}.
\begin{equation}
e = \lim_{n\to\infty} \left(1+\frac{1}{n}\right)^n
\label{Eq1}
\end{equation} 

Aligned \cref{Eq2}.
\begin{align}
	x^2 -25 &= 0 \nonumber \\
	x^2 &= 25 \nonumber \\
	\therefore \Aboxed{x &= 5}
 \label{Eq2}
\end{align}

Multi-line \cref{Eq3}.
\begin{multline}
	f(x) = 4xy^2 + 3x^3 - 2xy + 25x^3y^3 + 3xy - 4x^6y^6 + 3xy^2 - 2y^3\\ + a^3b^3c^4 + 3x^3 - 2xy + 25x^3y^3 + 3xy\\ - 4x^6y^6 + 3xy^2 - 2y^3 + a^3b^3c^4
\label{Eq3}
\end{multline}



% ---------------------------------
% Results
% ---------------------------------
\section{Results}
Here explain the results obtained from the experimental analysis.

Use table (\cref{table: style 2}) or figure (\cref{fig: style 2 image a} - \cref{fig: style 2 image c}) to explain the results. 

% Table with style 2
\begin{table}[ht]
\rowcolors{2}{gray!10}{white}
\centering
\caption{A table without vertical lines.}
\begin{tabular}[t]{ccccc}
\toprule
\textbf{Column 1}&\textbf{Column 2}&\textbf{Column 3}&\textbf{Column 4}&\textbf{Column 5}\\
\midrule
Entry 1&1&2&3&4\\
Entry 2&1&2&3&4\\
Entry 3&1&2&3&4\\
Entry 4&1&2&3&4\\
\bottomrule
\end{tabular}
\label{table: style 2}
\end{table}

% Three different figures
\begin{figure}[H]
\centering
\begin{minipage}{0.3\textwidth}
  \centering
  \includegraphics[width=\textwidth]{example-image-a}
  \captionof{figure}{image a}
  \label{fig: style 2 image a}
\end{minipage}
\hfill
\begin{minipage}{0.3\textwidth}
  \centering
  \includegraphics[width=\textwidth]{example-image-b}
  \captionof{figure}{image b}
  \label{fig: style 2 image b}
\end{minipage}
\hfill
\begin{minipage}{0.3\textwidth}
  \centering
  \includegraphics[width=\textwidth]{example-image-c}
  \captionof{figure}{image c}
  \label{fig: style 2 image c}
\end{minipage}
\end{figure}

% ---------------------------------
% Discussion
% ---------------------------------
\section{Discussion}
Discuss the outcomes here. What surprising finding you have made, and why it is surprising?


% ---------------------------------
% Limitations
% ---------------------------------
\section{Limitations}
Here you can describe the experiment's limitations. You can list them point-wise.

\begin{enumerate}
    \item Item 1
    \item Item 2
    \begin{enumerate}
        \item subitem 1  
        \item subsubitem 1
        \item sub subitem 2
    \end{enumerate}
    \item Item 3
\end{enumerate}